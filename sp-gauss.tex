\setcounter{section}{0}
\setcounter{subsection}{0}

\chapter{高斯过程}

\section{计算相关函数和功率谱密度}

\subsection{\hyperref[A2010-2]{【2010-2】}}\label{Q2010-2}

设 $X(t)$ 为零均值宽平稳高斯过程,相关函数为 $R(\tau)$。$\theta$ 是 $[0, 2\pi]$ 上的均匀分布随机变量,且与 $X(t)$ 独立,$\omega_c$ 为常数,试计算
$$
Y(t)=\cos(\omega_ct+\theta+X(t))
$$
的相关函数。

\subsection{\hyperref[A2009-7]{【2009-7】}}\label{Q2009-7}

设 $\omega$ 服从 $N(\mu, \sigma^2)$,$\theta$ 服从 $[0, 2\pi]$ 的均匀分布,两者互相独立。试计算随机过程 $X(t)=\cos(\omega t+\theta)$ 的相关函数和功率谱密度。

\subsection{\hyperref[A2007-6]{【2007-6】}}\label{Q2007-6}

设零均值宽平稳高斯过程 $X(t)$ 的功率谱密度为 $S_X(\omega)=\frac{1}{\omega^2 + 1}$,请计算 $Y(t)=e^{\alpha X(t)}$ 的相关函数。

\subsection{\hyperref[A2014-5]{【2014-5】}}\label{Q2014-5}

设 $X(s)$ 为零均值高斯白噪声,$E(X^2(t))=\sigma^2$,考虑
$$
Y(t)=\int_0^tX(s)ds, \ Z(t)=\sin^3(Y(t))
$$
试计算 $Z(t)$ 的相关函数。

\subsection{\hyperref[A2014-3]{【2014-3】}}\label{Q2014-3}

设 $X, Y$ 是两个独立的实随机变量,$X$ 服从标准正态分布,$Y$ 服从参数为 1 的指数分布。设
$$
Z(t)=Xe^{jYt}, \ -\infty<t<\infty
$$
试问 $\{Z(t)\}$ 是否宽平稳?若是,求其自相关函数 $R_Z(\tau)$,功率谱密度 $S_Z(\omega)$。
\\\\
\section{计算条件分布}

\subsection{\hyperref[A2010-7]{【2010-7】}}\label{Q2010-7}

设 $X$ 和 $Y$ 为服从联合高斯分布的一维随机变量,方差分别为 $\sigma_X^2$ 和 $\sigma_Y^2$,相关系数为 $\rho$。设 $U=Y^3, V=Y^2$,试计算条件概率密度 $f_{X|U}(x|u)$ 和 $f_{X|V}(x|v)$。

\subsection{\hyperref[A2008-4]{【2008-4】}}\label{Q2008-4}

考虑零均值宽平稳高斯过程 $X(t)$,相关函数为 $e^{-\alpha|\tau|}$,设 T 为确定的时间常数,试求 $E(X^4(T)|X(0))$。

\subsection{\hyperref[A2014-4]{【2014-4】}}\label{Q2014-4}

设 $\{X(t),\ -\infty<t<\infty\}$ 是零均值高斯过程,自相关函数为 $R_X(\tau)=5\cos(\frac{\pi\tau}{2})3^{-|\tau|}$,试求

(1)$E\left[(X(3))^2|(X(2)+X(4))\right]=?$

(2)$E\left[(X(2)+X(3))|(X(2)+X(4))\right]=?$
\\\\
\section{线性滤波器设计}

\subsection{\hyperref[A2007-7]{【2007-7】}}\label{Q2007-7}

考虑零均值宽平稳 Gaussian 白噪声 $X(t)$,对于给定的$\Delta t$,请设计一款线性时不变滤波器(给出其传递函数),使得 $X(t)$ 通过该滤波器后得到的随机过程 
 $Y(t)$ 满足:$Y(t)$ 的采样过程$\{Y_n=Y(n\Delta t),n\in N\}$仍然是 Gaussian 白噪声。 

\subsection{\hyperref[A2009-8]{【2009-8】}}\label{Q2009-8}
设 $X(t)$ 是零均值高斯白嗓声,功率密度为$N_0/2$。试设计一款线性滤波器,使得 $X(t)$ 通过该滤波器后的输出 $Y(t)$ 满足
$$
E(Y(1)Y(3)|Y(2))=CY^2(2)
$$
其中C是确定性常数。
\\\\
\section{坐标变换} 
\subsection{\hyperref[A2009-2]{【2009-2】}}\label{Q2009-2} 
设 $(X_1, X_2)$ 为服从联合高斯分布的随机变量,均值均为 0 ,方差均为 1 ,相关系数为 $\rho$。如果将 $X_1$ 和 $X_2$ 用极坐标进行表示
$$
X_1=R\cos(\phi), \ X_2=R\sin(\phi)
$$
试计算 $\phi$ 的密度函数,并利用该密度,计算 $P(X_1X_2>0)$。

\subsection{\hyperref[A2008-3]{【2008-3】}}\label{Q2008-3}

令 $X,Y$ 为独立的 Gaussian 分布随机变量,均值分别为$m_1, m_2$,方差均为 1 ,试求 $\sqrt{X^2+Y^2}$ 的概率密度。
\\\\
\section{其他}

\subsection{\hyperref[A2010-8]{【2010-8】}去相关}\label{Q2010-8}

设 $X$ 和 $Y$ 为服从联合高斯分布的 n 维随机变量,协方差阵分别为 $\Sigma_X$ 和 $\Sigma_Y$,互协方差阵为 $\Sigma_{XY}$ 和 $\Sigma_{YX}$,试构造矩阵 $G$ 和 n 维随机变量 $V$ ,使得 $X=GY+V$,且满足 $V$ 与 $Y$ 独立(请给出 $V$ 的密度的解析表达式和 $G$ 的具体形式)。

\subsection{\hyperref[A2009-1]{【2009-1】}计算均值和相关}\label{Q2009-1}

设 $(X_1, X_2)$ 为服从联合高斯分布的随机变量,均值均为 0,方差均为 1,相关
系数为 $\rho$。如果
$$
\max_{c_1^2+c_2^2=1}E(c_1X_1+c_2X_2)^2=1
$$
其中 $c_1$ 和 $c_2$ 为实数,试计算 $\rho$ 以及 $E(X_1^2X_2^2)$。

\subsection{\hyperref[A2007-5]{【2007-5】}计算特征函数}\label{Q2007-5}

设 $U$ 是 $[0, 2\pi]$ 均匀分布的随机变量,随机变量 $X$ 独立于 $U$,且密度为
$$
f_X(x)=|x|^3e^{-\frac{x^4}{2}}, \ x\in\mathbb{R}
$$
设随机过程 $Y(t)=X^2\cos(\omega t+U)$,计算 $Y(t_1), Y(t_2), \dots, Y(t_n))$ 的 n 维特征函数。