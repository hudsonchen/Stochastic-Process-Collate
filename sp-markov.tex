\setcounter{section}{0}
\setcounter{subsection}{0}

\chapter{马尔科夫过程}

\section{计算平稳分布}

\subsection{\hyperref[A2007-2]{【2007-2】}进出问题之穿鞋} \label{Q2007-2}

每天早上张三都要出门跑步,张三的家有前后两个门,门口都有一些鞋,两个门口鞋的数目之和为 $N$。张三出门时,如果门口有鞋则穿上,如果没有,就只好光脚跑了;跑步回来进门时,如果穿着鞋,则将鞋脱下放在门口。假定张三出门时选择前后门的概率相同,回家时也同样,请计算充分长时间以后,张三出门时不幸要光脚跑步的概率。

\subsection{\hyperref[A2014-7]{【2014-7】}进出问题之打伞(请和上题比较)}\label{Q2014-7}

小李有 3 把雨伞,上午上班时有雨就带一把到办公室,下午下班时有雨就带一把回家(中午不回家),其他情况不带雨伞。假设上下班时是否有雨是相互独立的,有雨的概率为 $p$。


(1)试定义一个马氏链来计算充分长时间后,小李会被雨淋的概率有多大?

(2)求出该马氏链的一步概率转移矩阵,判断各状态是否常返,并说明理由。

\subsection{\hyperref[A4-26]{【习题集4-26】}进出问题之打伞(仅做上题参考)}\label{Q4-26}

设某人有 $r$ 把伞,分别放在家里和办公室里,如果出门遇下雨(概率为 $p$,$0<p<1$),手边也有伞,他就带一把用;如果天晴他就不带伞。试证:经过相当长的一段时间后,这个人遇下雨但手边无伞可用的概率不超过 $\frac{1}{4r}$。
\\\\
\subsection{\hyperref[A2010-4]{【2010-4】}比赛问题之无吸收壁}\label{Q2010-4}

三名网球选手 $A,B,C$ 进行比赛,每一轮都是两人比赛,一人轮空,本轮比赛的胜者下一轮与本轮轮空的选手进行比赛。设三名选手的实力分别为 $S_A,S_B,S_C$,每次两人比赛时,选手 $X$ 击败选手 $Y$ 的概率为 $S_X/(S_X+S_Y)$,其中 $X,Y \in \{A,B,C\}$。试计算时间充分长后,各名选手实际参加比赛数目占总比赛数目的比例。

\subsection{\hyperref[A4-20]{【习题集 4-20】}比赛问题之有吸收壁(请和上题比较)}\label{Q4-20}

甲乙两人进行比赛,设每局比赛甲胜的概率为 $p$,乙胜的概率为 $q$,和局的概率为 $r$,$p+q+r=1$。设每局比赛后胜者获 1 分,负者获 -1 分,和局获 0 分。当两人中有一个人获得 2 分时,结束比赛。以 $X(n)$ 表示比赛至第 n 局时,甲获得的分数,$\{X(n), n=0,1,2,\dots\}$ 是一个齐次 Markov 链。


(1)写出此 Markov 链的状态空间;

(2)写出状态转移矩阵;

(3)计算二步转移概率矩阵;

(4)问在甲获得 1 分的情况下,再赛 2 局就结束比赛的概率为多少?
\\\\
\subsection{\hyperref[A2010-6]{【2010-6】}图上的随机游动}\label{Q2010-6}

有限简单非定向图由一些顶点和连接顶点的边构成,每条边连接两个不同顶点,每两个顶点间至多有一条边相连,没有孤立顶点。考察有限简单非定向图上的随机游动,当第 $n$ 时刻处在顶点 $i$ 上时,第 $n+1$ 时刻将跳转到与顶点 $i$ 有边直接连接的某个顶点 $j$ 上,转移概率为 $P(i, j)=1/d(i)$,其中 $d(i)$ 为与 $i$ 有边直接相连的顶点数目。试计算该随机游动的平稳分布。
\\\\
\subsection{\hyperref[A2009-5]{【2009-5】}自适应}\label{Q2009-5}

教师不断进行考试以督促学生学习,设考试有三种难度,易、中、难,学生在三种难度考核下答出好成绩的概率分别为 $0.9, \alpha, 0.1$。如果学生答出好成绩,教师在下次考试中就会提高难度,反之,会降低难度。如难度无法提高(降低)即保持不变。试计算,充分长时间后,如果教师希望学生们所经历的考试中,中等难度所占的比例不小于 $70\%$,那么应该怎样设置难度,即怎样设置 $\alpha$?

\subsection{\hyperref[A2007-1]{【2007-1】}循环}\label{Q2007-1}

设 $Y_n$ 是掷均匀的骰子 $n$ 次后得到的点数之和,请计算
$$
\lim_{n\rightarrow\infty} P(Y_n=0 \mod 13)
$$
其中 $Y_n=0 \mod 13$ 表示 $Y_n$ 可以被 13 整除。

\subsection{【2008-8】}

设 $\{X_n, n=0, 1, 2, \dots\}$ 为 Markov 链,一步转移概率矩阵为 $P$,令 $Y_n=(X_n, X_{n+1})$,很明显这也是 Markov 链。如果设 $\{X_n\}$ 的不变分布为 $\pi=(\pi_0, \pi_1, \dots)$,试求 $\{Y_n\}$ 的不变分布。
\\\\
\section{判断常返性}

\subsection{\hyperref[A2014-9]{【2014-9】}掷骰子}\label{Q2014-9}

同时掷 5 个骰子,将结果中出现次数最多的数字所对应的骰子固定住(例如,出现 23345,则将对应的 33 的二号和三号骰子固定住。如果出现两个以上数字,出现次数并列最多,则任取其中一个,并固定住其对应骰子);继续掷没有固定住的骰子,并将出现次数最多的数字所对应的骰子固定住(注意,允许数字有变化,例如上例中一、三、四号骰子继续掷,如果出现 43344,那么就固定住一、三、四号骰子)。考虑 Markov 链 $\{X_k\}$,状态空间为 $\{0, 1, 2, 3, 4, 5\}$。事件 $\{X_k=n\}$表示第 k 次抛掷后,出现次数最多的数字所出现的次数为 n。考察该链各状态的常返性。
\\\\
\section{计算分布 $\vec V_n$}

\subsection{\hyperref[A2008-7]{【2008-7】}抛硬币}\label{Q2008-7}

设两枚不均匀硬币分别编号为 1 和 2,抛掷硬币 1,正面向上的概率为 $p$;抛掷硬币 2,正面向上的概率也为 $p$。现开始如下抛掷过程:反复抛掷一枚硬币,直至出现反面向上,然后换为反复抛掷另一枚硬币,出现反面再换回来,如此循环往复。

(1)请计算:时间充分长之后,抛掷硬币 1 的概率。

(2)如果初始时刻抛掷的是硬币 1,请计算第 $5,6,7$ 及第 $10,11,12$ 次抛掷均抛掷硬币 2 的概率。

\subsection{\hyperref[A4-35]{【习题集 4-35】}数字传输系统}\label{Q4-35}

在传送数字 0 和 1 的通信系统中,传送每个数字必须经过若干级,而每一级中数字正确传送的概率为 $p$。设 $X(0)$ 表示进入系统的数字,$X(n)$ 表示离开系统第 $n$ 级的数字,$\{X(n), n=0, 1, 2, \dots\}$ 是齐次 Markov 链。

(1)写出状态转移概率矩阵;

(2)求出 $n$ 步转移概率矩阵;

(3)求平稳分布。
\\\\
\section{计算吸收概率}

\subsection{\hyperref[A4-34]{【习题集 4-34】}赌徒输光问题}\label{Q4-34}

赌徒甲有 $a$ 元,赌徒乙有 $b$ 元,两人进行赌博。每赌一局负者给胜者 1 元,没有和局,直到两人中有一个输光为止。设在每一局中甲胜的概率为 $\frac{1}{2}$,$X(n)$ 表示第 $n$ 局时甲的赌金,$\{X(n), n=0, 1, 2, \dots\}$ 为齐次 Markov 链。

(1)写出状态空间和状态转移概率矩阵;

(2)求出甲输光的概率。